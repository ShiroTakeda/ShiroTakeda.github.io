% Filename:            econ-sample.tex
% Author:              Shiro Takeda
% First-written:       <2002/11/02>
% Time-stamp:          <2011-12-04 14:04:35 Shiro Takeda>
%
% This document explains how to use econ.bst.
%
% $Id: econ-sample.tex,v 1.3 2008/12/22 09:10:00 st Exp $

%############################## Main #################################
\documentclass[10pt]{article}
%% Use natbib.sty.
\usepackage{natbib}
\usepackage{fancybox}
\usepackage{mathpazo}
\usepackage{url}
\usepackage{graphicx}
\usepackage{color}
\definecolor{MyBrown}{rgb}{0.3,0,0}
\definecolor{MyBlue}{rgb}{0,0,0.3}
\definecolor{MyRed}{rgb}{0.4,0,0.1}
\definecolor{MyGreen}{rgb}{0,0.4,0}
\usepackage[dvipdfm,
bookmarks=true,%
bookmarksnumbered=true,%
colorlinks=true,%
linkcolor=MyBlue,%
citecolor=MyRed,%
filecolor=MyBlue,%
urlcolor=MyGreen%
]{hyperref}

\setlength{\topmargin}{-15pt} 
\setlength{\textheight}{638pt}
\setlength{\oddsidemargin}{25pt}
\setlength{\textwidth}{420pt}

\makeatletter
\long\def\@footnotetext{%
  \insert\footins\bgroup
    \normalfont\footnotesize
    \interlinepenalty\interfootnotelinepenalty
    \splittopskip\footnotesep
    \splitmaxdepth \dp\strutbox \floatingpenalty \@MM
    \hsize\columnwidth \@parboxrestore
    \protected@edef\@currentlabel{%
       \csname p@footnote\endcsname\@thefnmark
    }%
    \color@begingroup
      \@makefntext{%
        \rule\z@\footnotesep\ignorespaces}%
      \futurelet\next\fo@t}
\def\fo@t{\ifcat\bgroup\noexpand\next \let\next\f@@t
                                \else \let\next\f@t\fi \next}
\def\f@@t{\bgroup\aftergroup\@foot\let\next}
\def\f@t#1{#1\@foot}
\def\@foot{\@finalstrut\strutbox\color@endgroup\egroup}

\newenvironment{Frame}%
{\setlength{\fboxsep}{15pt}
\setlength{\mylength}{\linewidth}%
\addtolength{\mylength}{-2\fboxsep}%
\addtolength{\mylength}{-2\fboxrule}%
\Sbox
\minipage{\mylength}%
\setlength{\abovedisplayskip}{0pt}%
\setlength{\belowdisplayskip}{0pt}%
}%
{\endminipage\endSbox
\[\fbox{\TheSbox}\]}

\makeatletter
\ifx\undefined\bysame
\newcommand{\bysameline}{\hskip.3em \leavevmode\rule[.5ex]{3em}{.3pt}\hskip0.5em}
\fi
\makeatother


\makeatother

%% Definition of \BibTeX command
\makeatletter
\def\BibTeX{{\rm B\kern-.05em{\sc i\kern-.025em b}\kern-.08em
    T\kern-.1667em\lower.7ex\hbox{E}\kern-.125emX}}
\makeatother

%%% title, author, acknowledgement, and date
\title{\texttt{econ.bst}:\\ \BibTeX{} style for economics\\ (for ver. 2.0.1)}
\author{Shiro Takeda\thanks{email: {\ttfamily shiro.takeda@gmail.com}.}}
\date{December 3, 2011}

%#####################################################################
%######################### Document Starts ###########################
%#####################################################################
\begin{document}

%%% Title
\maketitle

\newlength{\mylength}
\setlength{\fboxsep}{15pt}
\setlength{\mylength}{\linewidth}
\addtolength{\mylength}{-2\fboxsep}
\addtolength{\mylength}{-2\fboxrule}

%%% Table of contents.
\tableofcontents

% \setlength{\baselineskip}{17pt}

%########################## Text Starts ##############################

\section{Main features}

``\texttt{econ.bst}'' provids the following features:
\begin{Frame}
 \begin{itemize}
  \item The author-year type citation combined with ``\texttt{natbib.sty}''.
  \item Reference style used in economics papers.
  \item Highly customizable.  You can easily customize reference style as
       you wish.
 \end{itemize}
\end{Frame}

The third feature is the key characteristic of ``\texttt{econ.bst}''.

\section{How to use ``\texttt{econ.bst}''}

\begin{itemize}
 \item ``\texttt{econ.bst}'' requires ``\texttt{natbib.sty}''. 
       If ``\texttt{natbib.sty}'' is not installed in your \TeX{} system, you must
       install it first.
 \item Put ``\texttt{econ.bst}'' file somewhere under the directory
       ``\texttt{/texmf/bibtex/bst}''.
 \item You may need to change the character code of ``\texttt{econ.bst}''
       according to your \TeX system.
 \item See ``\texttt{econ-sample.tex}'' for an example tex file.
\end{itemize}


\section{Customization}

``\texttt{econ.bst}'' defines many functions which have names like
``\texttt{bst.xxx.yyy}''.  You can easily customize the reference style by
changing the contents of these functions.

\subsection{Notes on customization}

\begin{itemize}
 \item Customization here is customization of the reference part.  Style
       in the citation part mainly depends on a style file for citation
       (``\texttt{natbib.sty}'' or ``\texttt{harvard.sty}'').
 \item Except for some cases, customization here cannot change order
       of fields (order of author, year, title etc.)
 \item Functions with ``\texttt{.pre}'' define strings attached to the start
       of the field and functions with ``\texttt{.post}'' define strings
       attached to the end of field.  For exmple,
       ``\texttt{bst.author.pre}'' defines strings attached before author.
 \item You can change order of entries (references).  It will be
       explained in Section \ref{sec:sort_rule}.
\end{itemize}

\subsection{Examples of customization}

\subsubsection{Change delimiter used to separate mutiple author names
   from ``and'' to ``\&''.}

For this, change the contents of ``\texttt{bst.and}'' and ``\texttt{bst.ands}''.
\paragraph{Default definition:}
\begin{Frame}
\begin{verbatim}
FUNCTION {bst.and}
{ " and " }
FUNCTION {bst.ands}
{ ", and " }
\end{verbatim}
\end{Frame}

\paragraph{New definition:}
\begin{Frame}
\begin{verbatim}
FUNCTION {bst.and}
{ " \& " }
FUNCTION {bst.ands}
{ " \& " }
\end{verbatim}
\end{Frame}

Then, author names in reference part are displayed as follows:
\begin{center}
Fujita, Masahisa, Paul~R. Krugman, and Anthony~J. Venables \\
 $\downarrow$ \\
Fujita, Masahisa, Paul~R. Krugman \& Anthony~J. Venables 
\end{center}

See ``\verb|econ_b.bst|''.

\subsubsection{Make author to small caps style}

For this, change the contents of ``\texttt{bst.author.pre}'' and ``\texttt{bst.author.post}''.
\paragraph{Default definition:}
\begin{Frame}
\begin{verbatim}
FUNCTION {bst.author.pre}
{ "" }
FUNCTION {bst.author.post}
{ "" }
\end{verbatim}
\end{Frame}

\paragraph{New definition:}
\begin{Frame}
\begin{verbatim}
FUNCTION {bst.author.pre}
{ "\textsc{" }
FUNCTION {bst.author.post}
{ "}" }
\end{verbatim}
\end{Frame}

Then, author names in reference part are changed as follows:
\begin{center}
Fujita, Masahisa, Paul~R. Krugman, and Anthony~J. Venables \\
 $\downarrow$ \\
\textsc{Fujita, Masahisa, Paul~R. Krugman, and Anthony~J. Venables}
\end{center}

See ``\verb|econ_b.bst|''.

\subsubsection{Change the style of volume and number}

For this, change the contents of ``\texttt{bst.volume.pre}'',
``\texttt{bst.volume.post}'', ``\texttt{bst.number.pre}'' and
``\texttt{bst.number.post}''.  

\paragraph{Default definition:}
\begin{Frame}
\begin{verbatim}
FUNCTION {bst.volume.pre}
{ ", Vol. " }
FUNCTION {bst.volume.post}
{ "" }
FUNCTION {bst.number.pre}
{ ", No. " }
FUNCTION {bst.number.post}
{ "" }
\end{verbatim}
\end{Frame}

\paragraph{New definition:}
\begin{Frame}
 \begin{verbatim}
FUNCTION {bst.volume.pre}
{ ", \textbf{" }
FUNCTION {bst.volume.post}
{ "}" }
FUNCTION {bst.number.pre}
{ " (" }
FUNCTION {bst.number.post}
{ ")" }
\end{verbatim}
\end{Frame}

By this, the style of volume and number change from ``Vol. 5, No. 10''
to ``\textbf{5} (10)''.  See \verb|econ_b.bst|.


\subsubsection{Abbreviation of author name}

By default, when there are mutiple documents of the same author, author
name except for the first document is abbreviated by \verb|\bysame|
command (i.e. \bysameline).

If you want to always show author name for all documents, change the
content of ``\texttt{bst.use.bysame}'' as follows:
\begin{Frame}
\begin{verbatim}
FUNCTION {bst.use.bysame}
{ #0 }  
\end{verbatim}
\end{Frame}

In the default setting (`bst.use.bysame' is set to \#1), author names
are abbreviated when they are exactly the same. For example, suppose
that there are the following entries
\begin{itemize}
 \item Mazda, A., Subaru, B., and Honda, C., (2011) "ABC"
 \item Mazda, A., Subaru, B., and Honda, C., (2011) "DEF"
 \item Mazda, A., Subaru, B., and Toyota, D., (2011) "GHI"
\end{itemize}
\vspace*{1em}

In the default setting, these entries are listed like
\begin{Frame}
\begin{itemize}
 \item Mazda, A., Subaru, B., and Honda, C., (2011) "ABC" 
 \item \bysameline, (2011) "DEF"
 \item Mazda, A., Subaru, B., and Toyota, D., (2011) "GHI"
\end{itemize}
\end{Frame}

That is, the abbreviation of authors by `bysame' is only applied to
entries with exactly the same authors.

If you set \#2 to `bst.use.bysame' like
\begin{Frame}
\begin{verbatim}
FUNCTION {bst.use.bysame}
{ #2 }  
\end{verbatim}
\end{Frame}
you can choose alternative abbreviation style like
\begin{Frame}
\begin{itemize}
 \item Mazda, A., Subaru, B., Honda, C., (2011) "ABC"
 \item \bysameline, \bysameline, and \bysameline, (2011) "DEF"
 \item \bysameline, \bysameline, and Toyota, D., (2011) "GHI"
\end{itemize}
\end{Frame}
\vspace*{1em}

See ``\verb|econ_a.bst|'' and ``\verb|econ_b.bst|'' for examples.


\subsubsection{Order of first and last name in author field}

``\texttt{bst.author.name}'' defines order of first and last name in author field.

\paragraph{Default definition:}
\begin{Frame}
\begin{verbatim}
FUNCTION {bst.author.name}
{ #0 }
\end{verbatim}
\end{Frame}

If you change \verb|#0| to \verb|#1| or \verb|#2|, you can customize
order of first and family name.  For example, suppose author field is
defined as follows:
\begin{verbatim}
  author =	 {Masahisa Fujita and Paul R. Krugman and Anthony J. Venables}
\end{verbatim}

According to the content of ``\texttt{bst.author.name}'', expression of
author changes as follows:
\begin{Frame}
\begin{enumerate}
 \item \verb|#0|: First author $\rightarrow$ last-first, other authors
       $\rightarrow$ first-last.\\
       $\rightarrow$ Fujita, Masahisa, Paul~R. Krugman, and
       Anthony~J. Venables
 \item \verb|#1|: All authos $\rightarrow$ last-first \\
       $\rightarrow$ Fujita, Masahisa, Krugman, Paul~R., and Venables, Anthony~J.
 \item \verb|#2|: All authors $\rightarrow$ first-last \\
       $\rightarrow$ Masahisa Fujita, Paul~R. Krugman, and Anthony~J. Venables
\end{enumerate}
\end{Frame}

\subsubsection{First name in initial}

By default, first name is displayed in full.  If you change the content
of ``\texttt{bst.first.name.initial}'' to non-zero, first name is displayed
in initial.  For example,
\begin{center}
Fujita, Masahisa, Paul~R. Krugman, and Anthony~J. Venables \\
 $\downarrow$ \\
Fujita, M., P.~R. Krugman, and A.~J. Venables
\end{center}

See ``\verb|econ_a.bst|'' for an exmpale.

\subsubsection{Decapitalize letters in title field}

Suppose that the title field is defined as follows
\begin{center}
  \verb| title =	{Econometric Policy Evaluation: A Critique}|
\end{center}

Then, title is displayed in reference as follows:
\begin{center}
 Econometric Policy Evaluation: A Critique
\end{center}

If you change the content of ``\texttt{bst.title.lower.case}'' to
non-zero, letters except the first letter are decapitalized.  That is,
you get the following expression in reference:
\begin{center}
 Econometric policy evaluation: A critique
\end{center}

See ``\verb|econ_a.bst|'' for an exmpale.


\subsubsection{Number index before documents in reference}

You can put the number index to each documents as in
``\texttt{plain.bst}''.  For this, change the content of
``\texttt{bst.use.number.index}'' to non-zero.
\begin{Frame}
\begin{verbatim}
FUNCTION {bst.use.number.index}
 { #1 }
\end{verbatim}
\end{Frame}

If you use fonts other than computer modern fonts, you had better adjust
the contents of functions ``\texttt{bst.number.index.xxx.yyy}''.

See ``\verb|econ_b.bst|'' for an exmpale.


\subsubsection{List old references first}

By default, references written by the same author are listed in
chronological order (old documents are listed first).  If you change the
contents of ``\texttt{bst.reverse.year}'' to non-zero, the order is
reversed.
\begin{Frame}
\begin{verbatim}
FUNCTION {bst.reverse.year}
{ #1 }
\end{verbatim}
\end{Frame}

\subsubsection{Change the position of year}

By default, year is always displayed right after author name.  You can
change the place of year by setting other values to
``\texttt{bst.year.position}''.
\\

If non-zero set to ``\texttt{bst.year.position}'', year is placed
\begin{itemize}
 \item at the end of line if there is no ``\texttt{note}'' field, 
 \item and before ``\texttt{note}'' field if there is.
\end{itemize}
for non-article type entry.

With respect to aritcle type entry, the following rule is applied:
\begin{Frame}
\begin{enumerate}
 \item \verb|#1|: $\rightarrow$ year is placed at the end (before note field).
 \item \verb|#2|: $\rightarrow$ year is placed after journal name.
 \item \verb|#3|: $\rightarrow$ year is placed after volume.
\end{enumerate}
\end{Frame}

For example, reference style changes as follows:
\begin{description}
 \item[\#0: ] Mazda, A. and B. Subaru, 2007, ``ABC,'' \textit{Journal of Automobiles}, Vol. 1, pp. 1-10.
 \item[\#1: ] Mazda, A. and B. Subaru, ``ABC,'' \textit{Journal of Automobiles}, Vol. 1, pp. 1-10, 2007.
 \item[\#2: ] Mazda, A. and B. Subaru, ``ABC,'' \textit{Journal of Automobiles}, 2007, Vol. 1, pp. 1-10.
 \item[\#3: ] Mazda, A. and B. Subaru, ``ABC,'' \textit{Journal of Automobiles}, Vol. 1, 2007, pp. 1-10.
\end{description}

See ``\verb|econ_a.bst|'' and ``\verb|econ_b.bst|'' for exmpales.


\subsubsection{Order of editors and book title in incollection and inproceedings entry}

By default, editor name comes before book title in incollection and
inproceedings entry. You can reverse this order by
setting non-zero to `bst.editor.btitle.order' like
\begin{Frame}
\begin{verbatim}
FUNCTION {bst.editor.btitle.order}
{ #1 }
\end{verbatim}
\end{Frame}


\begin{Frame}
Krugman, Paul~R. (1991) ``Is Bilateralism Bad?'' in Elhanan Helpman and Assaf
  Razin eds.  {\it International Trade and Trade Policy}, Cambridge, MA: MIT
  Press, pp. 9--23.
\begin{center}
 $\downarrow$
\end{center} 
Krugman, Paul~R. (1991) ``Is Bilateralism Bad?'' in  {\it International Trade
and Trade Policy}  eds. by Elhanan Helpman and Assaf Razin, Cambridge, MA:
MIT Press, pp. 9--23.
\end{Frame}

\subsubsection{Order of address and publisher}

By default, publisher address is placed before publisher name.  You can
reverse this order by setting non-zero to `bst.address.position'.
\begin{Frame}
Krugman, Paul~R. (1991) ``Is Bilateralism Bad?'' in Elhanan Helpman and
  Assaf Razin eds.  {\it International Trade and Trade Policy},
  Cambridge, MA, MIT Press, pp. 9--23.
\begin{center}
 $\downarrow$
\end{center} 
Krugman, Paul~R. (1991) ``Is Bilateralism Bad?'' in Elhanan Helpman and Assaf
  Razin eds.   {\it International Trade and Trade Policy}, MIT Press,
  Cambridge, MA, pp. 9--23.
\end{Frame}


\section{Sorting rule}
\label{sec:sort_rule}

\noindent \textbf{[Note]} If you want to create an ordinary list of
references, you need not to read this part.  The explanation below is
for sorting references in a special way.

\subsection{Basic sorting rule}

The sorting of references is done according to values of fields defined
in bib files.  Basically, sorting is done according to the following
order of priority:
\begin{enumerate}
 \item Type of entry (if ``\texttt{bst.sort.entry.type}'' has non-zero value)
 \item Value of ``\texttt{year}'' field (if ``\texttt{bst.sort.year}'' has non-zero value)
 \item Value of ``\texttt{absorder}'' field.
 \item Value of ``\texttt{author}'' (or ``\texttt{editor}'') field.
 \item Value of ``\texttt{year}'' field.
 \item Value of ``\texttt{order}'' field.
 \item Value of ``\texttt{month}'' field.
 \item Value of ``\texttt{title}'' field.
\end{enumerate}

By default, 
\begin{itemize}
 \item ``\texttt{bst.sort.entry.type}'' and ``\texttt{bst.sort.year}'' have
       zero,
 \item \textbf{''''\texttt{absorder}''''} and \textbf{''''\texttt{order}''''}
       fields are not assigned values because they are fields specific
       to ``\texttt{econ.bst}''.
\end{itemize}
Thus, references are sorted according to
\begin{Frame}
 \begin{center}
 ``\texttt{author}'' $\rightarrow$ ``\texttt{year}'' $\rightarrow$
 ``\texttt{month}'' $\rightarrow$  ``\texttt{title}'' 
\end{center}
\end{Frame}
That is, ``\texttt{author}'' is used as the primary key, ``\texttt{year}'' as
the secondary key, ``\texttt{month}'' as the third key and ``\texttt{title}'' as
the fourth key.

\subsection{No sorting}

If you want to list references in citation order, set non-zero value to
``\texttt{bst.no.sort}''.
\begin{Frame}
\begin{verbatim}
FUNCTION {bst.no.sort}
{ #1 }
\end{verbatim}
\end{Frame}

Note that when you set non-zero value to ``\texttt{bst.no.sort}'', you had better not
use \verb|\bysame|.

\subsection{Sort references by type}

If you want to gather references according to their types (article,
book, incollection, unpublished etc.), set non-zero value to
``\texttt{bst.sort.entry.type}''.
\begin{Frame}
\begin{verbatim}
FUNCTION {bst.sort.entry.type}
{ #1 }
\end{verbatim}
\end{Frame}

Order of listing by entry type is determined by function
``\texttt{bst.sort.entry.type.order}'' (by default, listed in alphabetical
order, that is, article $\rightarrow$ book $\rightarrow$ booklet
$\rightarrow$ comment $\rightarrow$ inbook $\rightarrow$ incollection
$\rightarrow$ $\cdots$ $\rightarrow$ unpublished).  See
``\texttt{bst.sort.entry.type.order}'' in ``\texttt{econ.bst}''.

\subsection{Use ``\texttt{year}''  as the primary sorting key}

When you create CV or a list of your papers, you may want to sort
references in chronological order.  If all papers are written soley by
yourself, references are sorted in chronological order by default.
However, there are co-writers and if you are not the first author,
references are not sorted in chronological order because the author name
is used as the primary sorting key by default.
If you want to sort references in chronological order even when
there are co-writers, set non-zero to ``\texttt{bst.sort.year}''.
\begin{Frame}
\begin{verbatim}
FUNCTION {bst.sort.year}
{ #1 }
\end{verbatim}
\end{Frame}

By default, old references are listed first.  But if you set non-zero to
``\texttt{bst.reverse.year}'', new references are listed first.

\subsection{Sorting by ''\texttt{absorder}'' field}

If ``\texttt{absorder}'' is defined in bib file,
``\texttt{econ.bst}'' uses its content as the primary sorting key.
You can set number 0--999 to ''\texttt{absorder}'' field.

\begin{Frame}
\begin{center}
 no absorder or absorder = 0  $\rightarrow$ absorder = 1 $\rightarrow$ absorder = 2
 $\rightarrow$ $\cdots$ $\rightarrow$ absorder = 999
\end{center}
\end{Frame}
That is, reference with a small value of ``\texttt{absorder}'' is listed
first.  In this document (``\texttt{econ-sample.bib}''), the reference
with the key \citet{takeda10:_cge_analy_welfar_effec_trade} has 999 for
``\texttt{absorder}'' field and thus listed in the last.


\subsubsection{Ignore ''\texttt{absorder}'' field}

If you set some values for ``\texttt{absorder}'' fields in bib file,
but if you want to ignore them, set non-zero to
``\texttt{bst.notuse.absorder.field}''.
\begin{Frame}
\begin{verbatim}
FUNCTION {bst.notuse.absorder.field}
{ #1 }
\end{verbatim}
\end{Frame}

\section{Misc.}

\begin{itemize}
 \item Email: \verb|<shiro.takeda@gmail.com>|.
 \item \texttt{econ.bst} is available at \url{http://shirotakeda.org/home/tex/econ-bst.html}.
\end{itemize}
\vspace*{1em}

\citet{borgers95:_note_implem_stron_domin},
\citet{bergemann11:_ration},
\citet{takeda2011c},
\citet{Takeda2011b},
\citet{Takeda2011a},
\citet{Biker-2007-unemployment},
\citet{Babiker-1999-JapaneseNuclearPower},
\citet{Babiker-1999-KyotoProtocoland},
\citet{Babiker2000525},
\citet{BabikerRutherford-2005-EconomicEffectsof},
\citet{goldin:katz:2011}, \citet{goldin:katz:2008}, \citet{goldin:katz:2000}.
\vspace*{1em}

\nocite{*}

%%% BibTeX style.
\bibliographystyle{econ}

%% BibTeX database file.
\bibliography{econ-sample}

\end{document}

%#####################################################################
%######################### Document Ends #############################
%#####################################################################
% </pre></body></html>
% mode: yatex
% End:

