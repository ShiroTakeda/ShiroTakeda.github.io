%#!lualatex
%#BIBTEX biber test

%% クラスファイル = jlreq
\documentclass[article]{jlreq}

% biblatex の読み込み.
\usepackage[style=authoryear,nameorder,backref]{biblatex-japanese}

% % その他のbiblatex の option の指定
\ExecuteBibliographyOptions{
sortcites=true,%
date=year,              % date を year のみ表示
giveninits=true,        % Given name はイニシャルのみとする
maxbibnames = 8,	% 参考文献部分の著者数が 8 を越えたら省略
minbibnames = 3,        % その際、参考文献部分の著者を 3 名までに省略
maxcitenames = 2,       % 引用部分の著者数が 2 を越えたら省略
mincitenames = 1,       % その際、引用文献部分の著者を 1 名までに省略
uniquelist=minyear,     % 上の著者名省略の機能に対する制限
isbn=false,              % ISBN は非表示
}

% % localisation stringsの変更.これで書式を変更.
\DefineBibliographyStrings{japanese}{%
japanese-editor	 = {(編)},%
japanese-editors	 = {(編)},%
japanese-name-delimiter	 = {・},%
japanese-volume-prefix	 = {第},
japanese-volume-suffix	 = {巻},
japanese-number-prefix	 = {\addcomma{}第},
japanese-number-suffix	 = {号},
}

% ------------------------------------------------------------
%	同じ著者が続く場合の省略用の線の形式の変更
%
% authoryear style では同じ著者が続く場合に横線で省略される。その形式は
% \bibnamedash で定義されている。
% \bibnamedash is originally defined in biblatex.def.
\renewcommand{\bibnamedash}{%
  \hskip.2em \leavevmode\rule[.5ex]{2.5em}{.3pt}\hskip0.4em}
% 前の空白                    高さ 横の長さ 太さ 後ろの空白

% % 文献データベースの指定.
% \addbibresource{library.bib}

% 文献データベースの指定.
\addbibresource{test.bib}

\begin{document}

% 引用部分、参考文献部分は input-text.tex というファイル.

\section{文献の例}

\subsection{外国語文献}

\begin{itemize}
 \item \textbf{Article}: 
       \textcite{brezis93:_leap},
       \textcite{ishikawa94:cje},
       \textcite{Biker-2007-unemployment},
       \textcite{takeda12:cce},
       \textcite{takeda07:jjie},
       \textcite{yamazaki13:_japan},
       \textcite{takeda10:irae},
       \textcite{babiker05:ej},
       \textcite{yamasue09:mt},
       \textcite{yamasue07:mt},
       \textcite{babiker00:ep},
       \textcite{parry97:ree},
       \textcite{takeda19:ere},
       \textcite{takeda14:eeps},
       \textcite{imbens19:reas},
       \textcite{attwood06:_sexed_up},
       \textcite{aksin20063027},
       \textcite{baez/article},
       \textcite{bohringer07:ecoe},
       \textcite{bohringer06:ecoe},
       \textcite{krugman79:jie},
 \item \textbf{Book}:
       \textcite{krugman91:_geog},
       \textcite{helpman91:_int},
       \textcite{fujita99:_spatial},
       \textcite{ryza15:_advan},
       \textcite{pearl2009Causality},
       \textcite{attwood09:_mains_sex},
       \textcite{attwood10:_porn},
       \textcite{jones84:_hb_int},
       \textcite{jones85:_hb_int},
       \textcite{jones97:_hb_int},
       \textcite{aristotle:poetics},
       \textcite{aristotle:physics},
       \textcite{aristotle:anima}
 \item \textbf{Collection}:
       \textcite{westfahl:frontier},
 \item \textbf{Incollection}:
       \textcite{krugman91:_bila},
       \textcite{lucas76:_critique},
       \textcite{DeGorter2002},
       \textcite{balistreri20131513},
       \textcite{westfahl:space},
       \textcite{Mcconnell2005},
 \item \textbf{Unpublished}:
       \textcite{ishikawa03:_ghg},
       \textcite{rutherford00:_gtap},
       \textcite{takeda15:_lab},
       \textcite{babiker99:_kyoto},
 \item \textbf{Inbook}:
       \textcite{wong95:_int},
       \textcite{milne-thomson68:_theor},
       \textcite{kant:kpv},
       \textcite{kant:ku},
       \textcite{nietzsche:historie},
 \item \textbf{Inproceedings}:
       \textcite{wang89:_model},
       \textcite{zhang2016Deep},
 \item \textbf{Manual}:
       \textcite{brooke03:_gams},
       \textcite{cms},
 \item \textbf{Techreport}:
       \textcite{Peri2007},
 \item \textbf{Thesis}:
       \textcite{krugman77:_essay},
       \textcite{loh},
 \item \textbf{Patent}:
       \textcite{sorace},
 \item \textbf{Periodical}:
       \textcite{jcg},
 \item \textbf{Report}:
       \textcite{chiu},
 \item \textbf{Online}:
       \textcite{ctan},
\end{itemize}

\subsection{日本語文献}

\begin{itemize}
 \item \textbf{Article}:
       \textcite{iwamoto91jp:haito-keika},
       \textcite{Hattori00},
       \textcite{Hattori01},
       \textcite{Hattori02},
       \textcite{40018847518},
       \textcite{40017004376},
       \textcite{40020418914},
       \textcite{hosoda2017},
       \textcite{takeda2017300208},
       \textcite{120005614155},
       \textcite{120005678435},
 \item \textbf{Book}:
       \textcite{somusho04jp:2000io-kaisetsu},
       \textcite{imai71:_micr_1},
       \textcite{imai72:_micr_2},
       \textcite{ito85:_inte_trad},
       \textcite{kuroda97jp:keo},
       \textcite{miyazawa02:_io_intr},
       \textcite{katayama2001},
       \textcite{markusen99jp:trade_vol_1},
       \textcite{barro97jp},
       \textcite{nishimura90:_micr_econ},
       \textcite{arimura-takeda2012},
       \textcite{arimura2017jp},
       \textcite{a.___2000},
       \textcite{matloff__2012},
       \textcite{Boswell-2012},
       \textcite{Ryza2016},
       \textcite{ThoughtWorksinc.08},
       \textcite{chang-2013},
       \textcite{kuriyama20jp},
       \textcite{Mori-201206},
       \textcite{Uchida-90},
       \textcite{Yokomizo-2007},
       \textcite{Kuldell2018:_bio},
 \item \textbf{Incollection}:
       \textcite{oyama99:_mark_stru},
       \textcite{isikawa02jp:_env_trade},
       \textcite{takeda12:_cge},
 \item \textbf{Inbook}:
       \textcite{kiyono93:_regu_comp_1},
 \item \textbf{Online}:
       \textcite{takeda13:jecon},
 \item \textbf{Unpublished}:
       \textcite{naikakufu_2011},
       \textcite{takeda16jp:_gtap},
       \textcite{takeda07jp:cge},
 \item \textbf{PhdThesis}:
       \textcite{uzawa-phd-62},
\end{itemize}

% 以下は参考文献の表示

% 全部まとめて
\printbibliography[heading=bibintoc]

% タイプ別
\printbibliography[title={Articleタイプ},type=article]
\printbibliography[title={Bookタイプ},type=book]
\printbibliography[title={Incollectionタイプ},type=incollection]
\printbibliography[title={Inproceedingsタイプ},type=inproceedings]
\printbibliography[title={Inbookタイプ},type=inbook]
\printbibliography[title={Unpublishedタイプ},type=unpublished]
\printbibliography[title={その他のタイプ},nottype=article,nottype=book,nottype=incollection,nottype=inproceedings,nottype=inbook,nottype=unpublished]
        


%%% Reference 表示
\printbibliography[heading=bibintoc]

\end{document}

%#####################################################################
%######################### Document Ends #############################
%#####################################################################

 --------------------
 Local Variables:
 fill-column: 80
 coding: utf-8-dos
 End:
